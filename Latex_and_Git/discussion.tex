\section{Discussion}
\subsection{Emotion Recognition Performance}
Our collective results showed that individuals with higher social anxiety did not suffer a greater impairment in recognizing masked facial expressions. In our analysis, not only did we fail to find significant interaction effects between social anxiety level and mask use on hit rate, but we also observed that social anxiety was associated with higher hit rates for angry and fearful faces.
Along with a few studies using subclinical samples \citep{SUTTERBY2012242, tibi2011social}, the current results indicate that social anxiety, at subclinical levels, may rather be associated with an enhanced theory of mind, especially for negative emotions. When people with subclinical levels of social anxiety are more sensitive to others’ negative evaluations reflected by facial expressions, they may feel more self-conscious of their behavior during social interactions, causing more anxiety. On the contrary, SAD patients may suffer from an impairment in decoding social signals and experience difficulties in how to respond to them. Indeed, \citet{Nikolić} found a quadratic pattern of the Reading the Mind in the Eyes Test (RMET) performance and SA levels in children: at subclinical levels, social anxiety was associated with better RMET performance, which was mediated by blushing, a physiological marker of heightened self-consciousness. At clinical levels, however, social anxiety was associated with worse RMET performance. Future research should note the potential difference in theory of mind mechanisms underlying social anxiety in the clinical and subclinical populations, and examine it with other paradigms.
\subsection{Eye Movement Patterns}
In addition, our result was also inconsistent with our hypothesis regarding eye movement patterns. We did not find a significant interaction effect between social anxiety levels and mask use, nor did we find a main effect of SA levels on eye movement patterns, suggesting that social anxiety was not associated with a particular eye movement strategy. Therefore, our result is inconsistent with previous studies indicating that high SA individuals tended to avoid direct eye gaze compared with low SA counterparts \citep{horley2003social,weeks2019fear}. The heterogeneous attentional styles within the socially anxious population may explain our failure to uncover the singular eye movement pattern linked to social anxiety. Recent research suggests there may be heterogeneity of threat-related attentional processes among people with high levels of social anxiety. In particular, \citet{Chan16112020} used EMHMM to evaluate the eye movement patterns of healthy college students during free-viewing of neutral and angry faces. Those who consistently adopted either the eye-centered strategy (representing attentional vigilance towards threats) or the nose-centered strategy (representing attentional avoidance), when viewing both facial expressions tended to have higher levels of social anxiety. Hence, two distinct threat-related attentional biases—vigilance and avoidance—may exist among high SA individuals, which may mitigate our efforts to find a uniform eye movement pattern associated with social anxiety in the present study.



