\section{Introduction}
Social anxiety (SA) is characterized by the fear of being evaluated by others in social situations \citep{leary1997social}. The level of SA is considered to be a continuum, with social anxiety disorder (SAD) being at the most severe end. Patients with SAD suffer from intense distress in social situations and exhibit maladaptive avoidance behaviors, leading to significant functional impairment in daily life \citep{american2013diagnostic}.

Cognitive models have suggested the symptoms of social anxiety are attributed to impaired processing of social stimuli, especially facial expressions \citep{HEIMBERG2014705}. Facial expressions are important social signals that convey positive (e.g. acceptance) and negative evaluations (e.g. rejection) of others. Current literature has proposed several sociocognitive biases in people with high levels of SA that potentially impair facial expression processing, including (1) theory of mind deficit, and (2) eye gaze avoidance.

The COVID pandemic has aggravated social anxiety in the general population, as longitudinal studies showed significant increases in social anxiety levels among adolescents and adults in the community \citep{Hawes_2022}. At the same time, mask mandates may have introduced additional challenges for social interaction in the general population with a rising level of social anxiety, even though most may not reach the clinical threshold.

Hence, in an exploratory review, \citet{Saint03092021} have suggested cognitive bias modifications to mitigate the potential impact of masks on socially anxious individuals. However, evidence showing high SA individuals having greater impairments in recognizing emotions on masked faces has been limited, while some research has been done in populations with alexithymia \citep{bs14040343}.

Previous studies have suggested high SA individuals present eye gaze avoidance behavior. However, little is known about whether eye gaze avoidance would lead to a larger impairment of facial emotion recognition on masked faces. To examine this, the current study adopted the eye movement analysis with hidden Markov models approach (EMHMM). The EMHMM is a data-driven, machine-learning-based approach that provides a quantitative measure of eye movement patterns, which has been previously used in facial emotion recognition \citep{zheng2023differential} and single-face free-viewing tasks \citep{Chan16112020}. Using this approach, an individual’s eye movements were summarized using a hidden Markov model (HMM), including both person-specific regions of interest (ROIs) and transition probabilities among the ROIs. Subsequently, representative eye movement patterns among participants can be discovered through clustering individual HMMs, and dissimilarity between individual HMMs can be quantified by the log-likelihood of the individual’s data given the representative patterns. Previous studies have consistently discovered two representative face-viewing strategies: an “eye-centered” strategy that focuses more on the eye region, and a “nose-centered” strategy that focuses on the center of the faces more. In the context of social anxiety, previous research has interpreted the extent to which high SA individuals adopt the nose-centered strategy relative to the eye-centered strategy as their tendency to avoid direct eye gaze \citep{Chan16112020}. In other words, high SA individuals may be less likely to adopt an eye-centered strategy if they present eye gaze avoidance behavior.
\subsection{Current Study}
This study aimed to investigate emotion recognition performance of individuals with subclinical social anxiety when viewing masked faces and examine their eye movement patterns using hidden Markov models. Using an undergraduate sample, this study aimed to understand how masks interact with the sociocognitive mechanism in subclinical social anxiety and whether cognitive bias modifications suggested by researchers should be applied not only to SAD patients, but the larger subclinical population.

We made the following hypotheses:
\begin{enumerate}
\item A higher level of subclinical SA would be associated with a larger drop in hit rate in masked conditions, where reliance on the eye region was necessary for recognizing emotions.
\item A higher level of subclinical SA would be associated with a greater tendency to adopt a more nose-centered viewing strategy in both masked and unmasked conditions.
\end{enumerate}
