\documentclass[a4paper,man,natbib,floatsintext,12pt]{apa7}

\usepackage[english]{babel} %character and hyphenation rules specific to the language you choose
%\usepackage[utf8x]{inputenc}
\usepackage{graphicx}
\usepackage{color}
\usepackage{tikz}
\usepackage{amsmath}
\usepackage{blindtext}
\usepackage{tabularx} %great for APA-style Tables
\usepackage{siunitx} % Required for good table alignmen
\sisetup{
  round-mode          = places, % Rounds numbers
  round-precision     = 2, % to 2 places
}
\usepackage{multirow}
\usepackage{booktabs}
\usepackage{wrapfig}
\usetikzlibrary{shapes,decorations,arrows,calc,arrows.meta,fit,positioning}
\tikzset{
    -Latex,auto,node distance =1 cm and 1 cm,semithick,
    latent/.style ={ellipse, draw, minimum width = 0.7 cm},
    observed/.style ={rectangle, draw},
    bidirected/.style={Latex-Latex,dashed},
    el/.style = {inner sep=2pt, align=left, sloped}
}
\newcommand{\sigFtest}[4]{\textit{F}(#1,#2) = #3, \textit{p}$<$#4}
\newcommand{\nonsigFtest}[3]{\textit{F}(#1,#2) = #3, \textit{p}$>$.05}


\title{Impact of Mask Use on Facial Emotion Recognition in Individuals with Subclinical Social Anxiety: An Eye-tracking Study}
\shorttitle{SOCIAL ANXIETY AND MASKS ON FACIAL EMOTION RECOGNITION }
\author{Jackie Wai Yi Wo}
\affiliation{Department of Psychology, The University of Hong Kong, Hong Kong SAR, China}
\journal{Cognitive Research: Principles and Implications}
\abstract{Previous studies suggested that social anxiety is associated with theory of mind deficit and eye gaze avoidance when identifying facial emotions. We tested the hypothesis that socially anxious individuals would be more affected by mask use during facial emotion recognition. Eighty-eight healthy undergraduates with various levels of social anxiety were invited. Participants judged the emotions of masked and unmasked facial expressions. Eye Movement Analysis with Hidden Markov Models was used to analyze participants’ eye movement patterns during the task. Results failed to support our hypothesis. Instead, higher social anxiety was associated with higher hit rates for angry and fearful faces, regardless of mask use. Eye movement patterns were similar across social anxiety levels. Thus, our study indicates social anxiety, at least at subclinical levels, may be associated with a generally heightened sensitivity to negative emotions.}
\keywords{Social anxiety, Emotion recognition, Mask, Eye movement, EMHMM}
\authornote{The research presented in this assignment has been adapted from my published paper\citep{wo2025impact}. I acknowledge that the use of this material is solely for practicing LaTeX formatting, not for publishing original academic work.}
\leftheader{Alternate page header in man mode}

%-------------- END PREAMBLE  -------------------


\begin{document}

\maketitle  %Insert my APA style title page

\section{Introduction}

Your introduction goes here.  You might talk about previous work on cognitive deficits in schizophrenia \citep{brenner2014role}. Or you might want to explain how you used the Student \citet{student1908probable} \textit{t} distribution and present the formula you used to do significance testing for a correlation (see equation \ref{eq1}).
\begin{equation} \label{eq1}
t=\frac{r_{xy} \sqrt{n-2}}{ \sqrt{1-r^{2}_{xy}}}
\end{equation}

\blindtext


\section{Method}

Your method goes here!
\subsection{Participants}
Information about your participants goes here!
\subsection{Materials and Procedure}
Information about the details of your experiment goes here!

\section{Results}
\newcommand{\Ftest}[4]
{\textit{F}(#1,#2) = #3, \textit{p}$=$
#4}
\subsection{Hit Rate}
First, a significant main effect of LSAS showed that higher LSAS scores were associated with higher hit rates in general, \Ftest{1}{684}{4.74}{.030}. A significant interaction between LSAS and emotion further revealed that the higher hit rate associated with social anxiety was emotion-specific, \Ftest{3}{684}{4.62}{.003}. Specifically, the positive association between LSAS and hit rates was only shown for angry, \Ftest{1}{509.53}{11.88}{.001}, and fearful faces, \Ftest{1}{509.53}{5.97}{.015}, but not for happy, \textit{p}=.856, or sad faces, \textit{p}= .390. Second, a significant main effect of mask use was observed, \Ftest{1}{684}{103.07}{.001}. Participants generally identified facial emotions less correctly in masked conditions. There was also a significant interaction between mask use and emotion, \Ftest{3}{684}{15.23}{.001}. A post-hoc test with Bonferroni correction showed that the negative impact of mask use on hit rate was only shown in fearful, happy, and sad faces, \textit{p}s <.001, while angry faces had a similar hit rate in both the masked and unmasked conditions, \textit{p}=1.000. This suggested that participants mainly relied on the regions around the eyes to identify angry faces while recognizing other emotions required information in the lower face. Thirdly, a significant main effect of emotion was also revealed, \Ftest{3}{684}{55.35}{.001}, and the Bonferroni-corrected post-hoc test showed that hit rates significantly differed among four emotions, with the highest hit rates for happiness, then fear, followed by anger, and lastly sadness (happiness vs. fear: \textit{p}= .005; other pairwise comparisons: \textit{p}s< .001). Contrary to our hypothesis, there was neither a significant interaction between LSAS and mask use, \Ftest{1}{684}{0.00}{.998}, nor a LSAS x mask use x emotion interaction, \Ftest{3} {684}{0.35}{.789}.
\subsection{Eye Movement Patterns During Facial Emotion Recognition}
The two representative eye movement patterns (Pattern A and Pattern B) during facial emotion recognition found by clustering in EMHMM are shown in Figure~\ref{fig:gearhead}. The two patterns were significantly different, as the data from participants adopting Pattern A were more likely to be generated by Pattern A HMM than Pattern B HMM, \textit{t}(449) = 14.49, \textit{p}< .001, and vice versa for data from participants adopting Pattern B, \textit{t}(253) = 9.68, \textit{p}< .001. In Pattern A, a scan path mostly started with a fixation at a broad region that centered at the midpoint between two eyes and spread from the eyebrows and above the upper lip (red, 97\%), with small occasions to start at the forehead region (magenta, 2\%) or at a smaller region only covering the eye and the nose regions (blue, 1\%). After the fixation to the broad region (red), the path typically switched to narrower regions covering the nose and eye regions (red to blue, 47\%) or eye region only (red to green, 48\%), and then mostly remained at the same regions. In contrast, in Pattern B, the scan path typically started at a region that centered at the nose tip and spread from the brow ridge to the mouth region (red, 68\%), with probabilities of starting at a smaller region centered at the left cheek of the model (blue, 15\%; magenta, 12\%) or small occasions that start at a broad region that centered the mouth (green, 5\%). Participants tended to remain at the same fixation region after making their first fixation to the nose-tip-centered region (red to red, 100\%).  In sum, Pattern A represented a more eye-centered viewing strategy when identifying facial emotions, whereas Pattern B described a more nose-centered strategy. Thus, we termed Pattern A the “eye-centered strategy” and Pattern B the “nose-centered strategy”. Afterward, we examined whether social anxiety was associated with the differences in the tendency to use a more eye-centered or a nose-centered strategy across conditions (quantified by the A-B scale).
The mixed-effect model showed a significant main effect of mask use, \Ftest{1}{602}{1089.52}{.001}, with an overall increase in using eye-centered strategy associated with mask use. There was no significant main effect of LSAS, \Ftest{1}{82}{0.32}{.570}, or emotion, \Ftest{3}{602}{1.41}{.240}. There was also no significant interaction effect: the interaction between LSAS and mask use, \Ftest{1}{602}{3.47}{.063}; the emotion x LSAS interaction, \Ftest{3}{602}{0.17}{.918}; the emotion x mask use interaction, \Ftest{3}{602}{1.00}{.391}; the interaction between emotion, mask use, and LSAS, \Ftest{3}{602}{0.59}{.619}. This indicated that participants with different social anxiety levels adopted a similar eye movement strategy during emotion recognition across emotion and mask use conditions. 
\begin{figure}[h!] 
\caption{\label{fig:gearhead} The Two Representative Patterns Found in EMHMM}
\centering
\includegraphics[width=0.7\textwidth]{frog_1.jpg}
\caption*{Notes. (a) The representative hidden Markov model. The ellipses on the left show the ROIs as 2-D Gaussian emissions. The table on the right shows the priors (the probabilities that a fixation sequence starts from the ellipses) and the transition probabilities from one ROI to another in the sequence. (b) The actual assignment of fixations to each ROI.}
\end{figure}


\section{Discussion}
Because the Discussion section will often refer back to the results, it is useful to take advantage of cross-referencing (see Table~\ref{tab:RTmeans}). 

Sometimes the discussion section may even include additional figures.  If the figures are small enough and you don't want them to take up the full line, you can always use the \texttt{wrapfigure} environment.

\begin{wrapfigure}{l}{0.4\textwidth}     \centering       
\includegraphics[width=0.25\textwidth]{brain-lateral.png}
\caption{\label{fig:latbrain} Yet another figure caption here.}
\end{wrapfigure}

Figure~\ref{fig:gearhead} was also a great example for the Results right? But what if you have a small figure and you just want to wrap the text neatly around it?  Never fear....the ``wrapfig'' package is here! The wrapfig environment allows you to neatly wrap text around your figures just like the journals do. \blindtext



\bibliography{references.bib}

\end{document}